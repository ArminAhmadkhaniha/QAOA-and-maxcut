\documentclass{beamer}

% Presentation title and personal information
\title{Solving the MAX-cut Problem using QAOA}
\author{Armin Ahmadkhaniha\\Computer Science PhD program}


% Theme selection (e.g., Warsaw, AnnArbor, CambridgeUS)
\usetheme{Warsaw}
\usecolortheme{dolphin} % Set color theme
\setbeamertemplate{navigation symbols}{}
\setbeamertemplate{headline}{}
% Optional packages
\usepackage{amsmath, amssymb} % For math symbols
\usepackage{physics}          % For quantum notation if needed
\usepackage{graphicx}         % For including graphics
\usepackage{tikz}
\usetikzlibrary{positioning, arrows.meta}

% Custom settings for a cleaner appearance
\setbeamertemplate{footline}[frame number] % Page numbers
\setbeamertemplate{navigation symbols}{} 

\begin{document}

% Title slide
\begin{frame}
    \titlepage
\end{frame}

% Outline slide (optional)
\begin{frame}{Outline}
    \tableofcontents
\end{frame}

% Sections and slides for your presentation
\section{Prerequisite}
\begin{frame}{Quadratic Optimization Problem}
    \begin{columns}
        \column{0.5\textwidth}
        \begin{itemize}
            \item Optimization problems with quadratic objective function and linear and quadratic constraints
            
            
        \end{itemize}
   
    
    \column{0.5\textwidth}
    \textbf{minimize}  $x^T Q x + c^T x$ \\
    \hspace*{20pt} $Q \in \mathbb{R}^{n \times n}, \quad c \in \mathbb{R}^n$ \\
    \textbf{subject to} \\
     $Ax \leq b$ \\
     $x^T Q_i x + a_i^T x \leq r_i$ \\
     $l_j \leq x \leq u_j$
    \end{columns}
\end{frame}
\begin{frame}{QUBO}
    \begin{columns}
        \column{0.5\textwidth}
    \begin{itemize}
        \item Special case: Quadratic Unconstrained Binary Optimization (QUBO)
        \begin{enumerate}
        \item Quadratic objective function
        \item No variable constraints
        \item Binary optimization variables
    
        \end{enumerate}
        \end{itemize}

        \column{0.5\textwidth}
        \textbf{minimize}  $x^T Q x + c^T x$ \\
        $x \in \{0,1\}^n$, $Q \in \mathbb{R}^{n \times n}, \quad c \in \mathbb{R}^n$ \\
       
    
    \end{columns}
\end{frame}
\begin{frame}{what is MAX-cut?}
   \centering
    \includegraphics{maxcut.pdf}
\end{frame}
\section{MAX-cut as QUBO}
\begin{frame}{MAX-cut as QUBO}
    \begin{columns}
        % Left column
        \begin{column}{0.35\textwidth}
            \textbf{MaxCut}

            \vspace{0.55cm}
            \textbf{Weight matrix}
            \[
            w = \begin{pmatrix}
            0 & 1 & 2 & 3 \\
            1 & 0 & 3 & 4 \\
            2 & 3 & 0 & 1 \\
            3 & 4 & 1 & 0 \\
            \end{pmatrix}
            \]

            
            \textbf{Cost function}
            \[
            C(x) = \sum_{i,j=1}^{n} W_{ij} x_i (1 - x_j)
            \]
        \end{column}

        % Middle column (Graph)
        \begin{column}{0.40\textwidth}
            \centering
         \includegraphics[width=\textwidth]{maxweight.pdf}
        \end{column}

        
    \end{columns}
\end{frame} 
   \begin{frame}{MAX-cut as QUBO}
    \textbf{QUBO}

            \vspace{0.5cm}
            \textbf{QUBO matrix and vector}
            \[
            c_i = \sum_{j=1}^{n} W_{ij}, \quad Q_{ij} = -W_{ij}
            \]

            \vspace{0.5cm}
            \textbf{Cost function}
            \[
            C(x) = \sum_{i,j=1}^{n} x_i Q_{ij} x_j + \sum_{i=1}^{n} c_i x_i 
            \]
            \[
            = x^T Q x + c^T x
            \]
   
    
   
   \end{frame}
\section{What is the ansatz?}
\begin{frame}{Hamiltonian derivation}
    
    \[
        H_C |x\rangle = C(x) |x\rangle
        \]
    
        \vspace{1cm}
        
        \textbf{QUBO cost function}
        \[
        C(x) = \sum_{i,j=1}^{n} x_i Q_{ij} x_j + \sum_{i=1}^{n} c_i x_i
        \]
    
        \vspace{1cm}
    
        \textbf{Hamiltonian operator}
        \[
        H_C = \sum_{i,j=1}^{n} \frac{1}{4} Q_{ij} Z_i Z_j - \sum_{i=1}^{n} \frac{1}{2} \left( c_i + \sum_{j=1}^{n} Q_{ij} \right) Z_i + \left( \sum_{i,j=1}^{n} \frac{Q_{ij}}{4} + \sum_{i=1}^{n} \frac{c_i}{2} \right)
        \] 
    
\end{frame}
\begin{frame}{Matrix exponential}
    \textbf{Matrix exponential} \\
    For a matrix $\mathbf{M}$, the matrix exponential is defined as the power series:
    \[
        e^{\mathbf{M}} = \sum_{k=0}^{\infty} \frac{1}{k!} \mathbf{M}^k
    \]
    
    \vspace{1cm}
    
    \[
        e^{-i \theta X} = R_X(2 \theta) = \begin{pmatrix} \cos(\theta) & -i \sin(\theta) \\ -i \sin(\theta) & \cos(\theta) \end{pmatrix}
    \]
    
    \[
        e^{-i \theta Y} = R_Y(2 \theta) = \begin{pmatrix} \cos(\theta) & -\sin(\theta) \\ \sin(\theta) & \cos(\theta) \end{pmatrix}
    \]
    
    \[
        e^{-i \theta Z} = R_Z(2 \theta) = \begin{pmatrix} e^{-i \theta} & 0 \\ 0 & e^{i \theta} \end{pmatrix}
    \]
    
    \vspace{1cm}
    
    \textbf{Pauli matrices}
    \[
        X = \begin{pmatrix} 0 & 1 \\ 1 & 0 \end{pmatrix}, \quad Y = \begin{pmatrix} 0 & -i \\ i & 0 \end{pmatrix}, \quad Z = \begin{pmatrix} 1 & 0 \\ 0 & -1 \end{pmatrix}
    \]
\end{frame}
\begin{frame}{Quantum Circuit}
   \textbf{Cost Layer}
   
    \[
H_C = \sum_{i,j=1}^n \frac{1}{4} Q_{ij} Z_i Z_j - \sum_{i=1}^n \frac{1}{2} \left( c_i + \sum_{j=1}^n Q_{ij} \right) Z_i
\]

\[
e^{-i \gamma H_C} = \prod_{i,j=1}^n R_{Z_i Z_j} \left( \frac{1}{2} Q_{ij} \gamma \right) \prod_{i=1}^n R_{Z_i} \left( \left( c_i + \sum_{j=1}^n Q_{ij} \right) \gamma \right)
\]

    

\end{frame}
\begin{frame}{Quantum Circuit}
    
 
    \includegraphics[width=\textwidth]{cost.pdf}
 
 \end{frame}
\begin{frame}{Quantum Circuit}
    
    \begin{columns}
        
        \begin{column}{0.6\textwidth}
            \textbf{Mixer Layer}

            \[
H_M = \sum_{i=1}^n X_i \quad \quad X = \begin{pmatrix} 1 & 0 \\ 0 & -1 \end{pmatrix}
\]

\[
e^{-i \beta H_M} = \prod_{i=1}^n R_X(2 \beta)
\]
        \end{column}

        % Middle column (Graph)
        \begin{column}{0.40\textwidth}
            
         \includegraphics[width=\textwidth]{mixer.pdf}
        \end{column}

        
    \end{columns}
    
 
     
 
 \end{frame}
\section{Codes}
                \begin{frame}{Codes}
                \textbf{Show us some coding!!!!}
                \url{https://github.com/ArminAhmadkhaniha/QAOA-and-maxcut/blob/main/qaoamax.ipynb}
                \end{frame}
                
\section{References}
\begin{frame}{References}
    \begin{itemize}
        \item  E. Farhi, J. Goldstone, and S. Gutmann, A Quantum Approximate Optimization Algorithm, ArXiv:1411.4028 [Quant-Ph] (2014)
    \end{itemize}
\end{frame}

\begin{frame}
    \begin{center}
        \textbf{Thank you for your time and attention!}
        \vspace{1cm}
        
        Questions and Discussion
        \vspace{1cm}
        
        \textbf{Contact Information:}
        \begin{itemize}
            %\item \textbf{Name:} Armin Ahmadkhaniha
            \item \textbf{Email:} ahmadkha@mcmaster.ca
            
        \end{itemize}
    \end{center}
\end{frame}


\end{document}
